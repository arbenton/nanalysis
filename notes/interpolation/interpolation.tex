\documentclass[10pt]{article}
\usepackage[margin=1in]{geometry}
\usepackage[T1]{fontenc}
\usepackage{amsmath}
\usepackage{amsfonts}
\usepackage{listings}
\usepackage{color}
\definecolor{mygreen}{RGB}{28,172,0} % color values Red, Green, Blue
\definecolor{mylilas}{RGB}{170,55,241}
\setlength\parindent{0pt}
\numberwithin{equation}{subsection}

\lstset{language=Matlab,%
    %basicstyle=\color{red},
    breaklines=true,%
    morekeywords={matlab2tikz},
    keywordstyle=\color{blue},%
    morekeywords=[2]{1}, keywordstyle=[2]{\color{black}},
    identifierstyle=\color{black},%
    stringstyle=\color{mylilas},
    commentstyle=\color{mygreen},%
    showstringspaces=false,%without this there will be a symbol in the places where there is a space
    numbers=left,%
    numberstyle={\tiny \color{black}},% size of the numbers
    numbersep=9pt, % this defines how far the numbers are from the text
    emph=[1]{for,end,break},emphstyle=[1]\color{red}, %some words to emphasise
    %emph=[2]{word1,word2}, emphstyle=[2]{style},
}

\begin{document}


    \section{Interpolation and Polynomial Approximation}

        Equations which interpolate given a set of points.

        \subsection{Lagrange Polynomials}

            Given a set of points $x$, and the output of their unknown function
            $f$, an interpolating function $P$ of order $n$ can be constructed.
            $$P(x) = \sum_{k=0}^n \Big[ f(x_k) \prod _{i=0, i\neq k}^n \frac{x - x_{i}}{x_{k} - x_{i}} \Big]$$

            Lagrange Polynomials are effective for the general case, but prone
            to round-off error.\\

            \lstinputlisting{../../code/matlab/interpolation/lagrange.m}

        \subsection{Neville's Method}

            Given a set of points $x$, and the output of their unknown function
            $f$, an interpolating polynomial can be recursively constructed.
            $$Q_{i,j} = \frac{(x - x_{i-j})Q_{i,j-1} - (x - x_i)Q_{i-1,j-1}}{x_i - x_{i-j}}$$
            Neville's Method reduces the computations required for interpolants
            and therefore reduces the round-off error from lagrange.\\

            \lstinputlisting{../../code/matlab/interpolation/neville.m}

        \subsection{Newton Divided Differences}

            Given a set of points $x$ and the output of their
            unkown function $f$. A divided difference formula is recursively
            applied.
            $$ f[x_i, ..., x_{i+k}] = \frac{f[x_{i+1},...,x_{i+k}] - f[x_{i},...,x_{i+k-1}]}{x_{i+k} - x_i}$$
            These functions are then used to find the interpolating polynomial
            $P_n$.
            $$P_n(x) = f[x_0] + \sum_{k=1}^n \Big[ f[x_0,...,x_k] \prod_{i=0}^{k-1} (x - x_i) \Big]$$
            Divided Differences can have additional points given to it without
            recalculation of the polynomial.\\

            \lstinputlisting{../../code/matlab/interpolation/divided_differences.m}

        \subsection{Newton Equally Spaced Differences}

            Given a set of equally spaced points $x$ and the output of their
            unkown function $f$, ordered in increasing order. A forward
            difference formula is recursively applied.
            $$P_n(x) = f(x_0) + \sum_{k=1}^n {s \choose k } \Delta^k f(x_0)$$
            If the equally spaced points are in decreasing order, then backward
            differences should be used.
            $$P_n(x) = f[x_n] + \sum_{k=1}^n (-1)^k {-s \choose k } \Delta^k f(x_n)$$
            The Divided Difference algorithm can then be applied.

        \subsection{Hermite's Method}

            \textit{Hermite Polynomials $H$ agree on the value of the function and
            its derivative at the points given.}\\

            Given a set of points $x$ and the values of their function $f$ and
            its derivative $f'$. Using divided differences and creating virtual
            nodes $z$ to represent the derivative values, the function can be
            interpolated.\\

            \lstinputlisting{../../code/matlab/interpolation/hermite.m}

        \subsection{Natural Cubic Splines}

            \textit{Splines break the domain into piecewise portions, using
            a different polynomial for each portion. They must be continuous
            to the second derivative across pieces.}\\

            Given a set of points $x$ and the values of their function $f$. A
            spline $S_j$ is constructed.
            $$S_j(x) = a_j + b_j(x-x_j) + c_j(x-x_j)^2 + d_j(x-x_j)^3$$
            A natural spline will be linear at its bounds $a, b$. That is,
            $S''(a) = 0$ and $S'(b) = 0$. A system of equations is constructed
            and solved.\\

            \lstinputlisting{../../code/matlab/interpolation/natural_cubic_splines.m}

        \subsection{Clamped Cubic Splines}

            Given a set of points $x$, the values of their function $f$, and
            the values of their first derivative at the endpoints. A
            spline $S_j$ is constructed. A spline is constructed with the
            additional endpoint constraint.

            \lstinputlisting{../../code/matlab/interpolation/clamped_cubic_splines.m}

\end{document}
