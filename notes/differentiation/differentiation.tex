
\documentclass[10pt]{article}
\usepackage[margin=1in]{geometry}
\usepackage[T1]{fontenc}
\usepackage{amsmath}
\usepackage{amsfonts}
\usepackage{listings}
\usepackage{color}
\definecolor{mygreen}{RGB}{28,172,0} % color values Red, Green, Blue
\definecolor{mylilas}{RGB}{170,55,241}
\setlength\parindent{0pt}
\numberwithin{equation}{subsection}

\lstset{language=Matlab,%
    %basicstyle=\color{red},
    breaklines=true,%
    morekeywords={matlab2tikz},
    keywordstyle=\color{blue},%
    morekeywords=[2]{1}, keywordstyle=[2]{\color{black}},
    identifierstyle=\color{black},%
    stringstyle=\color{mylilas},
    commentstyle=\color{mygreen},%
    showstringspaces=false,%without this there will be a symbol in the places where there is a space
    numbers=left,%
    numberstyle={\tiny \color{black}},% size of the numbers
    numbersep=9pt, % this defines how far the numbers are from the text
    emph=[1]{for,end,break},emphstyle=[1]\color{red}, %some words to emphasise
    %emph=[2]{word1,word2}, emphstyle=[2]{style},
}

\begin{document}


    \section{Differentiation}

        Algorithms which approximate $f'(p)$ given $f(x)$ and $p$.

        \subsection{Midpoint Formulas}

            Given a function $f(x)$ and point $p$, a the derivative can
            be found by considering the equally spaced points $p_i = p + i*h$,
            around $p$. Midpoint formulas consider the points both greater
            and lesser than point $p$, and can use 3 or 5 points.\\

            For functions with $f^{(3)}(x) = 0$, 3 point formulas are ideal.

            $$
                f'(x) = \frac{f(p + h) - f(p - h)}{2h} + \mathcal{O}(h^2)
            $$

            For functions with $f^{(5)}(x) = 0$, 5 point formulas are ideal.

            $$
                f'(x) = \frac{f(p-2h)-8f(p-h)+8f(p+h)-f(p+2h)}{12h} + \mathcal{O}(h^4)
            $$

            Midpoint derivatives are prone to round-off error, and therefore
            have an optimal non-zero stepsize $h$.

            \lstinputlisting{../../code/matlab/differentiation/midpoint_diff.m}

        \subsection{Endpoint Formulas}

            Given a function $f(x)$ and point $p$, a the derivative can
            be found by considering the equally spaced points $p_i = p + i*h$,
            around $p$. Endpoint formulas consider the points both greater
            than point $p$, and can use 3 or 5 points.\\

            For functions with $f^{(3)}(x) = 0$, 3 point formulas are ideal.

            $$
                f'(x) = \frac{-3f(p)+4f(p+h)-f(p+2h)}{2h} + \mathcal{O}(h^2)
            $$

            For functions with $f^{(5)}(x) = 0$, 5 point formulas are ideal.

            $$
                f'(x) = \frac{-25f(p)+48f(p+h)-36f(p+2h)+16f(p+3h)-3f(p+4h)}{12h} + \mathcal{O}(h^4)
            $$

            Endpoint derivatives are prone to round-off error, and therefore
            have an optimal non-zero stepsize $h$.

\end{document}
