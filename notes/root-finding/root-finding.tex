\documentclass[10pt]{article}
\usepackage[margin=1in]{geometry}
\usepackage[T1]{fontenc}
\usepackage{amsmath}
\usepackage{amsfonts}
\usepackage{listings}
\usepackage{color}
\definecolor{mygreen}{RGB}{28,172,0} % color values Red, Green, Blue
\definecolor{mylilas}{RGB}{170,55,241}
\setlength\parindent{0pt}
\numberwithin{equation}{subsection}

\lstset{language=Matlab,%
    %basicstyle=\color{red},
    breaklines=true,%
    morekeywords={matlab2tikz},
    keywordstyle=\color{blue},%
    morekeywords=[2]{1}, keywordstyle=[2]{\color{black}},
    identifierstyle=\color{black},%
    stringstyle=\color{mylilas},
    commentstyle=\color{mygreen},%
    showstringspaces=false,%without this there will be a symbol in the places where there is a space
    numbers=left,%
    numberstyle={\tiny \color{black}},% size of the numbers
    numbersep=9pt, % this defines how far the numbers are from the text
    emph=[1]{for,end,break},emphstyle=[1]\color{red}, %some words to emphasise
    %emph=[2]{word1,word2}, emphstyle=[2]{style},
}
\setlength\parindent{0pt}
\numberwithin{equation}{subsection}
\begin{document}

    \section{Root Finding}

        Solutions to the equation $f(x) = 0$ in one dimension.

        \subsection{Bisection Method}

            Given an interval $[a, b]$ containing a root, the interval is
            interatively halved. The interval contains a root if $f(a)f(b)<0$.\\

            The Bisection method always converges, so long as a root exists
            on the inteval given. The bisection method converges linearly,
            making it one of the slower options.

            \lstinputlisting{../../code/matlab/root-finding/bisection.m}

        \subsection{Fixed Point Iteration}

            \textit{A fixed point is the solution to $f(p) = p$. One is not guaranteed
            to exist.}\\

            Given a starting point $p$, a function $g(x) = x - f(x)$ is
            constructed, and its series is iterated.
            $$x_{i} = g(x_{i-1})$$
            The solution for $p$ exists where the solution converges. Fixed
            Point Iteration converges if $g$ is continuous on its range, and
            its range contains a fixed point.

            \lstinputlisting{../../code/matlab/root-finding/fixed_point.m}

        \subsection{Newton's Method}

            Given a starting point $p$, and the function $f'$, its series is
            iterated.
            $$x_{i} = x_{i-1} - \frac{f(x_{i-1})}{f'(x_{i-1})}$$
            Fixed Point iteration is then applied. Newton's Method converges
            quadratically, but requires the derivative
            of the function and the initial guess to be close to the root.

            \lstinputlisting{../../code/matlab/root-finding/newton.m}

        \subsection{Aitken's $\Delta^2$ Method}

            \textit{The $\Delta^k$ represents the $k$-order finite-derivative and is
            defined such that $\Delta p_n = p_{n+1} - p_{n}$ and
            $\Delta^k p_n = \Delta (\Delta^{k-1} p_n)$.}\\

            Using the first- and second-order finite-derivates, a series can be
            constructed.
            $$\hat{x}_n = x_n - \frac{(\Delta x_n)^2}{\Delta^2 x_n}$$
            This series can then be applied with fixed point iteration.
            Aitken's $\Delta^2$ Method generally converges much faster than the original
            series.

        \subsection{Horner's Method}

            Given that function $P$ is a polynomial with $n$ real roots,
            and polynomial $Q$ with no real roots, function $P$ can be
            factorized by $Q$.
            $$P(x) = Q(x) \prod_{i=0}^{n} (x - x_i) + a_0$$

            Horner's Method is rapid for polynomials and can find all zeros
            through recursion. Programmatically, the polynomial is represented
            as a coefficient array.

\end{document}
