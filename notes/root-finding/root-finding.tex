\documentclass[12pt]{article}
\usepackage[margin=1in]{geometry}
\usepackage[T1]{fontenc}
\usepackage{amsmath}
\usepackage{amsfonts}
\usepackage{algorithm}
\usepackage[noend]{algpseudocode}
\setlength\parindent{0pt}
\numberwithin{equation}{subsection}
\begin{document}

    \section{Root Finding}

        Solutions to the equation $f(x) = 0$ in one dimension.

        \subsection{Bisection Method}

            Given an interval $[a, b]$ containing a root, the interval is
            interatively halved. The interval contains a root if $f(a)f(b)<0$.\\

            The Bisection method always converges, so long as a root exists
            on the inteval given. The bisection method converges linearly,
            making it one of the slower options.

            \begin{center}
            \begin{algorithm}[H]
                \caption{The Bisection Method}
                \begin{algorithmic}[1]
                    \Procedure{FindRoot}{function $f$, interval $[a, b]$}
                    \While {$|f(p)| > \epsilon_{mach}$}
                        \State $p := (a + b) / 2$
                        \If{$f(p)f(a) > 0$}
                            \State $a := p$
                        \Else
                            \State $b := p$
                        \EndIf
                    \EndWhile
                    \State \Return $p$
                    \EndProcedure
                \end{algorithmic}
            \end{algorithm}
            \end{center}

        \subsection{Fixed Point Iteration}

            \textit{A fixed point is the solution to $f(p) = p$. One is not guaranteed
            to exist.}\\

            Given a starting point $p$, a function $g(x) = x - f(x)$ is
            constructed, and its series is iterated.
            $$x_{i} = g(x_{i-1})$$
            The solution for $p$ exists where the solution converges. Fixed
            Point Iteration converges if $g$ is continuous on its range, and
            its range contains a fixed point.

            \begin{center}
            \begin{algorithm}[H]
                \caption{Fixed Point Iteration}
                \begin{algorithmic}[1]
                    \Procedure{FindRoot}{function $f$, guess $p$}
                    \While {$|f(p) - p| > \epsilon_{mach}$}
                        \State $p \gets f(p)$
                    \EndWhile
                    \State \Return $p$
                    \EndProcedure
                \end{algorithmic}
            \end{algorithm}
            \end{center}

        \subsection{Newton's Method}

            Given a starting point $p$, and the function $f'$, its series is
            iterated.
            $$x_{i} = x_{i-1} - \frac{f(x_{i-1})}{f'(x_{i-1})}$$
            Fixed Point iteration is then applied. Newton's Method converges
            quadratically, but requires the derivative
            of the function and the initial guess to be close to the root.

            \begin{center}
            \begin{algorithm}[H]
                \caption{Newton's Method}
                \begin{algorithmic}[1]
                    \Procedure{FindRoot}{function $f$, derivative $f'$, guess $p$}
                    \While {$|f(p) - p| > \epsilon_{mach}$}
                        \State $p \gets p - f(p)/f'(p)$
                    \EndWhile
                    \State \Return $p$
                    \EndProcedure
                \end{algorithmic}
            \end{algorithm}
            \end{center}

        \subsection{Aitken's $\Delta^2$ Method}

            \textit{The $\Delta^k$ represents the $k$-order finite-derivative and is
            defined such that $\Delta p_n = p_{n+1} - p_{n}$ and
            $\Delta^k p_n = \Delta (\Delta^{k-1} p_n)$.}\\

            Using the first- and second-order finite-derivates, a series can be
            constructed.
            $$\hat{x}_n = x_n - \frac{(\Delta x_n)^2}{\Delta^2 x_n}$$
            This series can then be applied with fixed point iteration.
            Aitken's $\Delta^2$ Method generally converges much faster than the original
            series.

        \subsection{Horner's Method}

            Given that function $P$ is a polynomial with $n$ real roots,
            and polynomial $Q$ with no real roots, function $P$ can be
            factorized by $Q$.
            $$P(x) = Q(x) \prod_{i=0}^{n} (x - x_i) + a_0$$

            Horner's Method is rapid for polynomials and can find all zeros
            through recursion. Programmatically, the polynomial is represented
            as a coefficient array.

            \begin{center}
            \begin{algorithm}[H]
                \caption{Horners Method}
                \begin{algorithmic}[1]
                    \Procedure{FindRoot}{coefficents $a$, point $x$}
                    \State $y := a_i$
                    \State $n := ||a|| $
                    \For {$j \in \mathbb{Z}[n-1, 1]$}
                        \State $y \gets xy + a_j$
                    \EndFor
                    \State $y \gets xy + a_{0}$
                    \State \Return $p$
                    \EndProcedure
                \end{algorithmic}
            \end{algorithm}
            \end{center}

\end{document}
